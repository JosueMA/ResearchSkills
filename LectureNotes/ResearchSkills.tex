\documentclass{tufte-book}

\usepackage{color}
\usepackage{xcolor}
\usepackage{framed}
\usepackage{listings}

\usepackage{multicol}              
\usepackage{multirow}
\usepackage{booktabs} 

\usepackage[]{hyperref}
\definecolor{darkblue}{rgb}{0,0,.5}
\hypersetup{colorlinks=true, breaklinks=true, linkcolor=darkblue, menucolor=darkblue, urlcolor=darkblue, citecolor=darkblue}

\lstset{
language=[LaTeX]{TeX},
breaklines = true,
breakautoindent = false,
breakindent = 0pt,
commentstyle=\color{gray},
frame=single,
framerule=0.4pt,
framesep=3pt,
xleftmargin=3.4pt,
xrightmargin=3.4pt,
basicstyle=\normalfont,
keywordstyle=\color{blue}\sffamily,                                
identifierstyle=\color{black},
numbersep=5mm, 
numbers=left, 
numberstyle=\tiny,
morekeywords = {
maketitle, 
tableofcontents}  
}

\setcounter{secnumdepth}{1}


\title{Research skills - an introduction to the crafts of a scientist}
\author{Florian Hartig, University of Freiburg}

\begin{document}
\let\cleardoublepage\clearpage
\maketitle
\newpage
\tableofcontents

\newpage

\section*{Preface}

Word in progress. Please send comments to \href{http://florianhartig.wordpress.com/}{http://florianhartig.wordpress.com/}


\chapter{Logic and philosophy of science}


\section{Progress in small steps, or a history of science}

A common saying is that history repeats itself. In science, that is not strictly 


\section{The scientific method}

\subsection{The scientific method - official version}


\subsection{The scientific method - b-side}



\chapter{Finding a good question}


\section{Literature - standing on the shoulders of giants}

in the following chapters, we'll speak about logic, empiricism 


\chapter{Logic and rationale thinking}

In one of the foundational works of modern science, "Le Discours de la M�thode (1637)"  \footnote{Full title "Discourse on the Method of Rightly Conducting One's Reason and of Seeking Truth" \citep{Descartes-DiscourseMethodRightly-1673}}, Ren� Descartes famously observed: 

\begin{quote}
Good sense is, of all things among men, the most equally distributed; for every one thinks himself so abundantly provided with it, that those even who are the most difficult to satisfy in everything else, do not usually desire a larger measure of this quality than they already possess.
\end{quote}

These words of one of the founding fathers of what we today consider Science are often read and cited as a sarcastic observation of human vanity. However, 

\begin{quote}
And in this it is not likely that all are mistaken the conviction is rather to be held as testifying that the power of judging aright and of distinguishing truth from error, which is properly what is called good sense or reason, is by nature equal in all men; and that the diversity of our opinions, consequently, does not arise from some being endowed with a larger share of reason than others, but solely from this, that we conduct our thoughts along different ways, and do not fix our attention on the same objects. 
\end{quote}

\begin{quote}
For to be possessed of a vigorous mind is not enough; the prime requisite is rightly to apply it. The greatest minds, as they are capable of the highest excellences, are open likewise to the greatest aberrations; and those who travel very slowly may yet make far greater progress, provided they keep always to the straight road, than those who, while they run, forsake it. 
\end{quote}


\section{Logical fallacies}





\chapter{Empricism and empirical work}

\section{A statistical reminder}

\section{Design philosophies}

\section{Validity}

\section{Reproducibility}

\section{Statistical analysis}



\chapter{Communication}



Advice on writing essays and theses

Preparation ? Know what you want to say. Sounds trivial, but really, ?If you don?t know where you are going,you?ll end up someplace else? (Yogi Berra). Before you pile up words, try to explain a friend in 5 min the point of your theses. If you fail with this, think again about your story.

Start writing early ? Because, while writing, you will find flaws in your logic that you didn?t see before, and you can correct your direction. ?The time to begin writing an article is when you have finished it to your satisfaction. By that time you begin to clearly and logically perceive what it is that you really want to say.? (Mark Twain)

Mindset ? Make your goal, above all, clarity of thought and expression, and impervious logic of your argument. Google and read Woodford, F. (1967) Sounder thinking through clearer writing. Science, 156, 743.

Modesty ? Think of your essay or thesis as a piece of craftsmanship. You are an apprentice and you are asked to make a chair. What is expected of you is a simple piece of work that respects the rules you have been taught, and that is, above all, solid. You are not expected to add fancy ornamentation that might be used by an ingenious master craftsmen, and you are not asked to produce a entire set of furniture that would suffice to fit out a house. You would gain admiration and praise if you could manage to achieve one of the latter, but likely you will fail to even produce the very thing you were asked for while trying ? a simple, solid chair. Bottomline ? stay modest in what you want to achieve, but stay ambitions in how you achieve it!

Style ? Writing is difficult. Good writing is incredibly hard. If you think writing is no problem for you, you probably haven?t even realized the extent of your writing problems. Read

    My lecture notes on scientific writing (I don?t have then linked here, but I?ll send you a link upon request). The second part is about style. The first part discusses in detail expectations for the different parts (introduction, methods, discussion) or research papers. Respect conventions.
    This blog post by Brian McGill. The most important point of this post in my opinion: ?The battle for good writing is won sentence by sentence! A good sentence is: short, has the subject and verb together, has an active verb, has the points of emphasis at the beginning and end, and moves the reader along from a familiar launch point at the start to the new information at the end.?
    Writing tips by Mark Twain. Especially this one: ?As to the Adjective: when in doubt, strike it out.?
 \citep{Gopen-ScienceOfScientific-1990}

Disclaimer ? Different people / disciplines have different opinions about writing. You can?t ?prove? the correctness of style, so be adviced that my suggestions are no natural laws and can be overruled by you as wells as your supervisor or the editor of whatever journal you submit your work.



\section{Principles of communication}

\section{Oral presentation}

\section{Written }


\chapter{Scientific visualisation}


Kelleher-Tenguidelineseffective-2011


\chapter{Scientific writing}

\section{The purpose and the art of writing}

\section{Style}

\href{http://www.economist.com/styleguide/introduction}{http://www.economist.com/styleguide/introduction}

%\href{http://www.nature.com/authors/author_resources/how_write.html}{http://www.nature.com/authors/author_resources/how_write.html}



\section{Choice of words and common mistakes}

\subsection{Too complicated}

The data \textbf{featured/comprised} a pixel size of 10 (had)

A \textit{subsequent} ANOVA \textit{analysis} \textbf{enabled a quantification of} the impacts of the varied factors (quantified)

There are \textit{a number of records} in the literature \textit{focusing on comparisons} between \textit{sets of }modeling approaches while predicting biomass at plot scale (A number of previous studies has compared modeling approaches to predict biomass at the plot scale)


The results of our second experiment suggest - our results suggest

the \textit{explicit} findings of our two experiments


\subsection{Not neccessary}

As such, the data consisted ?.
A total number of 9 samples was 

\subsection{Logic and clarity}

However, yet, still, check if really necessary, and if 

\paragraph{Which and that} Which and that is another common point of confusion. Let's say that we have 10 samples. Compare the two sentences: 1) the samples, which were expose to radiation the day before, were analysed. 2) The samples that were exposed to radiation the day before were analyzed. There is a big difference between the two sentences. 1) says that all samples were exposed to radiation and then analysed 2) implies that not all samples were exposed to radiation, and only those that were were analysed. In grammar, 1) is called a non-restrictive clause, and 2) is called a restrictive clause. "that" always implies a restrictive meaning. We could leave it there, i.e. that 1) Which is non-restrictive, and 2) that is restrictive, and that used the be the rule, but the problem is that the use of "which" for restrictive clauses has crept into the English language. Thus, we can also say 3) The samples which were exposed to radiation the day before were analyzed. Notice that there is a tiny difference to 1) - there is no comma before which. This is the only indicator that allows us to know whether the restrictive or the non-restrictive meaning is implied. Many people are not aware of this difference though. In prose, the distinction may matter little, but for science, my opinion is that clarity goes before style, and I therefore recommend to strictly stay with 1) which for non-restrictive and 2) that for the restrictive meaning.



\subsection{Positive presentation}

Another common problem are defensive, 

Here, we aim / this study tries  ; There is no try, do it. Better: the objective of this study was to  / we examine / ...  

It seems that ; sounds as if you are not certain at all. OK if this is on a subject on which one can only specualate. Not OK if you could find out more.

To the best of our knowledge, no previous study ...  ; This is a borderline case. You will see this expression in scientific articles, and it's OK to use it. However, it does seem insecure. Use it only if you are really not sure, and if you have no means to make sure that what you are saying is correct. 



\chapter{A working scientists}

\section{Good scientific practice}

The term "good scientific practice"

\section{Social aspects}

\subsection{Collaborations}

\subsection{Conflicts}

\subsection{Teams}


\section{Science as a career}


\bibliographystyle{/Users/Florian/Home/Bibliography/Databases/bibstyles/Plos2009}
\bibliography{/Users/Florian/Home/Bibliography/Databases/flo}



\end{document}