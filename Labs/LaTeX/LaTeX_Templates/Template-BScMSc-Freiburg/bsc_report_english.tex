% This is a template designed by Maarten J. Waterloo for BSc and MSc
% students at the Faculty of Earth and Life Sciences, Vrije Universiteit
% Amsterdam, adapted to the University of Freiburg by Carsten F. Dormann. 

\documentclass[11pt,twoside,a4paper,final]{report}
% We define a two-side report on A-4 paper in final quality using a
% point size of 11 for the text. Other possibilities are {book},
% {article}, etc.

% A % sign indicates that text is commented out and will not appear in
% the document. Use \% if you want to indicate a % symbol in the text,
% thus 10\% will become 10%...

% We need to include some packages for layout and style. We do this
% below using the \usepackage command. A short explanation for each
% package is included.

\usepackage{amsmath}   % If you are going beyond the most basic level
                       % of displayed equations, you will benefit from
                       % using the amsmath package that plugs into
                       % LaTeX. This package has lots of useful
                       % features for multi-line equations, compound
                       % symbols, even commutative diagrams! Allows
                       % use \text in equations	

\usepackage{booktabs}  % booktabs is to enable the easy production of
                       % tables such as should appear in published
                       % scientific books and journals. What
                       % distinguishes these from plain LaTeX tables
                       % is the default use of additional space above
                       % and below rules, and rules of
                       % varying`thickness'.

\usepackage[top=3cm, bottom=3cm, outer=2.5cm, inner=3.5cm]{geometry}  
					   % formating the page margins

\usepackage{graphicx}  % The graphicx package implements LaTeX
                       % support for including graphics files,
                       % rotating parts of a page, and scaling parts
                       % of a page. The package depends on having a
                       % DVI driver that can produce these effects.
                       
\usepackage[hidelinks]{hyperref} %This package will make links in PDF
                       % documents to your figures, references, etc.
                       % hidelinks removes the boxes around the hyperreferenced items
                       
\usepackage[utf8]{inputenc} % Hard to explain: Define here the text encoding 
					   % used by your computer; on Linux typically utf8, on 
					   % windows ansinew and on mac often latin1;
					   % ideally, the whole world would use utf8
					   % important for öüäß (Umlaute)

\usepackage{libertine} % Use a nice serif font! Alternatives:
					   % times, palatino, ... 
					   
\usepackage{makeidx} % use this package to make an index
                       % of keywords, you have to follow this by the
                       % \makeindex command 
\makeindex             % Create the index file        

\usepackage{natbib}    % Provides a style with author-year and
                       % numbered references, as well as much detailed
                       % of support for other bibliography
                       % use. Provides versions of the standard BibTeX
                       % styles that are compatible with natbib,
                       % plainnat, unsrtnat,
                       % abbrnat. The bibliography styles produced by
                       % custom-bib are designed from the start to be
                       % compatible with natbib.
                       
\usepackage{rotating}  % The rotating package implements three
                       % environments within which in-line figures,
                       % tables, and captions can be rotated by an
                       % arbitrary number of degrees. Two additional
                       % environments allow rotation of floating
                       % objects, which are typeset alone on separate
                       % pages.

\usepackage{wrapfig}   % allows text to flow around a figure               


% The \raggedbottom declaration makes all pages the height of the text
% on that page. No extra vertical space is added.
\raggedbottom

%----------------------------------------------------------------------------
% Allow more floating material on text pages:
\renewcommand\floatpagefraction{.9}%
\renewcommand\topfraction{.9}%
\renewcommand\bottomfraction{.9}%
\renewcommand\textfraction{.1}%
\setcounter{bottomnumber}{4}%
\setcounter{topnumber}{4}%
\setcounter{totalnumber}{4}%


%----------------------------------------------------------------------------
				% here the content starts !		
%----------------------------------------------------------------------------
% Define title and authors
\def\maintitle{On the intersection of desirability, reachability and sleeping disorder: the role of the final thesis in the maturation of a student}
%\def\subtitle{Subtitle...}
\def\authors{Whathave I. Learned }
\def\supervisor{Supervisor: Prof. Dr. Irene Knowitall, Department of Irreproducible Affairs}
\def\cosupervisor{Co-supervisor: Prof. Dr. Jörg Noinput, Department for the Investigation of the Unknown}
\def\year{Freiburg, May 2014}
\def\course{Bachelor thesis  (Student ID 450100) \\submitted to \\the Faculty of Environment \& Natural Resources \\ at the Albert-Ludwigs-University Freiburg}

% Start the document
\begin{document}

% Make some shortcuts for writing units and chemical species
% You can add your own here...
\newcommand{\mgl}{mg l$^{-1}$}
\newcommand{\mmoll}{mmol l$^{-1}$}
\newcommand{\meql}{meq l$^{-1}$}
\newcommand{\muscm}{$\mu$S cm$^{-1}$}
\newcommand{\water}{H$_2$O}
\newcommand{\kion}{K$^+$}
\newcommand{\naion}{Na$^+$}
\newcommand{\caion}{Ca$^{2+}$}
\newcommand{\mgion}{Mg$^{2+}$}
\newcommand{\clion}{Cl$^-$}
\newcommand{\nitrate}{NO$_3^-$}
\newcommand{\nitrite}{NO$_2^-$}
\newcommand{\nhion}{NH$_4^+$}
\newcommand{\hcoion}{HCO$_3^-$}
\newcommand{\hardness}{Ca$^{2+}$ + Mg$^{2+}$}

% Pagestyle of numbering, options are plain (Just a plain page number), empty (empty heads and feet - no page numbers), 
% headings	(puts running headings on each page), and myheadings (You specify what is to go in the heading with the 
% \markboth or the \markright commands.
% No numbering on title page, so we use empty here...
\pagestyle{empty}

%-- french title page (r) ------------------------------------------------
% Include the university-logo at the bottom of the page
\begin{figure}[h]
  \begin{flushright}
    \vspace{-2cm}
    \includegraphics[width=4cm]{images/ufcd-logo-e1-a4-color} %ALU_longlogo}
  \end{flushright}
\end{figure}

\begin{center}
  {\huge\bfseries\maintitle\par}
  \vskip 2em%
%  {\Large\bfseries\subtitle\par}
%  \vskip 2em%
  {\Large\authors\par}
   \vskip 2cm%
  {\course\par}
  \vskip 1em%
   \begin{figure}[h]
     \begin{center}
       \includegraphics[width=10cm]{images/bsc-fossil}
    \end{center}
  \end{figure}
 % \vskip 2em%
\end{center}
  {\supervisor \\  \cosupervisor \\
  \flushright \year \par} 

\newpage
% If you have a photo on the titlepage, you can add a short
% explanataion here 
Cover photograph: Example of a fossil in a reef. Source: somewhere from the internet ...

\newpage

% Page numbering, options are arabic (Arabic numerals), roman	(Lowercase Roman), Roman (Uppercase Roman)
% alph (lowercase letters) and Alph (Uppercase letters)
\pagenumbering{Roman}
\pagestyle{plain}

% Include table of contents
\tableofcontents

% Remove the % in front of the command below to include a list of tables:
%\listoftables

% Remove the % in front of the command below to include a list of figures:
%\listoffigures

\newpage
% Use arabic numbering throughout the text
\pagenumbering{arabic}

% Include an abstract of your report
\abstract{

Place your abstract text here

} 



% Start with the first chapter
\chapter{Introduction}
% With \label you can define a crossreference so that you can refer
% to this chapter later in the text by using Chapter~\ref{ch:background}
\label{ch:introduction}


Start your introduction here...

\chapter{Methods}
\label{ch:methods}
You may want to start with a few, brief sentences of what you actually did, before going into the details over the next sections.

\section{Site}
\section{Species}
\section{Statistical Analysis}

  Describe the methods and measurements that you used here



\chapter{Results}
\label{ch:results}

Describe your results here...


\chapter{Discussion}
\label{ch:discussion}

Try to write introduction and discussion in a way that the reader doesn't really require methods and results. To do so, you have to use the first (few) sentences of the discussion to recapitulate your findings.

Discuss your results here, ideally structured into various sections, of which one is:


\section{Conclusions}
\label{sec:conclusions}

  Your conclusions summarise your main findings as presented in the
  Discussion chapter. Put them here...



\chapter*{Acknowledgements}
\label{ch:Acknowledgements}
\addcontentsline{toc}{chapter}{Acknowledgements}

And of course include your acknowledgements here: my supervisor was always there for me, taught me so much and hence will receive eternal praise and gratefulness. Actually, I rarely saw him/her and that was just as well.



   
% Insert the references
\bibliographystyle{plainnat}
\bibliography{library/studlib}

% Anything behind this command will be treated as an appendix
\appendix

% Insert the first appendix
\chapter{This is Appendix 1...}
\label{app:appendix1}

Put appendix text here


% Insert the second appendix

% insert a second appendix
\chapter{This is appendix 2...}
\label{app:appendix2}

Put appendix text here...




% If you used the makeidx package to create an index then use the 
% following command to print the index
\printindex

\chapter*{Selbstständigkeitserklärung} % (in German!)

\vspace{2cm}

\section*{Erklärung}

Ich versichere hiermit, dass ich die vorliegende Arbeit ohne fremde Hilfe selbstständig verfasst und nur die angegebenen Quellen und Hilfsmittel benutzt habe. Wörtlich oder dem Sinn nach aus anderen Werken entnommene Stellen habe ich unter Angabe der Quellen kenntlich gemacht.

\medskip
\noindent (I hereby declare that I have composed this document unassistedly and that I only used the sources and devices I declared. Passages taken verbatim or in meaning from other sources are identified as such and the sources are acknowledged and cited.)

\vspace{2cm}

\noindent \year

\end{document}